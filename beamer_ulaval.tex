% 2013-11-01
% exemple de présentation Beamer
% contact : admin@mat.ulaval.ca
\documentclass{beamer}
% Document en français, césure, chapitres, etc
\usepackage[T1]{fontenc}
\usepackage[utf8x]{inputenc}
\usepackage[francais]{babel}
\uselanguage{French}
\languagepath{French}
% permet d'include des figures TikZ
\usepackage{pgf}
\usepackage{tikz}

% thème de la présentation : Darmstadt
% la matrice des thèmes Beamer se trouve à cette
% adresse : http://www.hartwork.org/beamer-theme-matrix/
\mode<presentation> {
\usetheme{Darmstadt}
% couleurs et logo de l'Université Laval
\setbeamercovered{transparent}
\definecolor{rouge}{RGB}{255,0,0}
\definecolor{or}{RGB}{255,204,0}
\definecolor{noir}{RGB}{0,0,0}
\setbeamercolor{structure}{fg=rouge,bg=or}
\setbeamercolor{titlelike}{fg=noir,bg=or}
\setbeamercolor{frametitle}{fg=rouge,bg=or}
\setbeamercolor{item}{fg=rouge,bg=or}
\logo{\includegraphics[width=1.5cm]{logo_UL.jpg}}
}

%%%%%%%%%%%%%%%%%%%%%%%%%%%%%%%%%%%
% Début de la présentation Beamer %
%%%%%%%%%%%%%%%%%%%%%%%%%%%%%%%%%%%

\title{Exemple de présentation Beamer}
\author{Philémon Siclone}
\institute{Département de mathématiques et de statistique (DMS)\\
http://www.mat.ulaval.ca
}

\date{\today}

\begin{document}

% page titre
\begin{frame}
\titlepage
\end{frame}

% plan de la présentation
\begin{frame}
\frametitle{Plan de la présentation}
\tableofcontents
\end{frame}

%%%%%%%%%%%%
% Partie 1 %
%%%%%%%%%%%%
\section{Introduction}
\begin{frame}
\frametitle{Commentaires généraux}
\begin{itemize}
 \item Beamer est l'outil de choix pour faire des présentations mathématiques.
        C'est l'équivalent (\alert{en mieux!}) de PowerPoint. Contrairement à ce dernier,
        les équations y sont lisibles...
 \item Pour écrire des mathématiques, on utilise la syntaxe de \LaTeX.
 \item On peut cliquer sur l'item ou sur les cercles qui apparaissent en haut de
        la page pour naviguer dans la présentation. Chaque item correspond à une section
        et chaque cercle à une sous-section.
 \item Cette introduction vous indiquera quelques principes de base: syntaxe des transparents,
       comment insérer une figure, etc.
 \item Un gros merci à \alert{Till Tantau} l'auteur de Beamer!
\end{itemize}
\end{frame}

%%%%%%%%%%%%
% Partie 2 %
%%%%%%%%%%%%
\section{Mathématiques}
\subsection{Faire des mathématiques dans Beamer}
% partie 2 : page 1
\begin{frame}
\frametitle{Des constantes}
Quelques constantes importantes.
\begin{align*}
\varphi &= \frac{1+\sqrt{5}}{2} = 1.618\,033\,9887\ldots \\[0.5cm]
e &= \lim_{n\to\infty} \left( 1 + \frac{1}{n} \right)^n \\[0.5cm]
\pi &= \sum_{i=0}^{\infty} \frac{1}{16^i}
      \left( \frac{4}{8i+1} - \frac{2}{8i+4}
    - \frac{1}{8i+5} - \frac{1}{8i+6} \right)
\end{align*}
\end{frame}

% partie 2 : page 2
\begin{frame}
\frametitle{Des équations lisibles...}
\begin{itemize}
\item<1> $ \cos^2\theta + \sin^2\theta = 1 $
\item<2> Un problème trivial d'un illustre inconnu nommé Fermat: $a^n  + b^n = c^n$
\item<3> Que les fluides incompressibles soient!
\[
\left\{
\begin{array}{rcl}

 \nabla{}\cdot  u & =& 0\\
\\
  \displaystyle  \left(\frac{\partial  u}{\partial t} + u \cdot \nabla{ u}\right)+\nabla{p} -  
  \frac{2}{Re}\nabla{} \cdot \big(\dot\gamma( u)\big) & =&  f \\
\end{array}
\right.
\]
\end{itemize}
\end{frame}

% partie 2 : page 3
\subsection{Encore des mathématiques!}
\begin{frame}
\frametitle{Lemmes, théorèmes et corollaires}
\begin{lemma}[Siclone]
\label{trivia}
Si $a=2$ alors $a^2 = 4$.
\end{lemma}

\begin{theorem}[Pythagore]
\label{th_pyt}
   Pour un triangle droit de côtés $a$, $b$ et d'hypothénuse $c$, on a:
\[
  a^2 + b^2 = c^2
\]
\end{theorem}
\begin{proof}
La preuve est laissée en exercice.
\end{proof}
Le théorème de Pythagore est l'un des plus célèbres résultats.
On sous-estime beaucoup l'importance du lemme de Siclone. 
\end{frame}

%%%%%%%%%%%%
% Partie 3 %
%%%%%%%%%%%%
\section{Laboratoires informatiques}
% partie 3 : page 1
\begin{frame}
\frametitle{Les laboratoires informatiques}
Le département de mathématiques et de statistique compte
trois laboratoires informatiques.\\
\begin{center}
\begin{tabular}{|c|c|} \hline
{\bf local} & {\bf \# d'ordinateurs} \\ \hline \hline
1069 & 26 \\ \hline
1073 & 7  \\ \hline
1448 & 10 \\ \hline
\end{tabular}
\end{center}
\end{frame}

% partie 3 : page 2
\begin{frame}
\frametitle{Leur utilisation}
\begin{itemize}
\item Laboratoire de 1er cycle, local 1069.
\begin{center}
\includegraphics[width=6cm]{1069.jpg}
\end{center}
\item Le laboratoire informatique, situé au local 1073, est réservé
      aux étudiants gradués.
\item Celui du 1448 est réservé aux chanceux du GIREF!
\end{itemize}
\end{frame}

%%%%%%%%%%%%
% Partie 4 %
%%%%%%%%%%%%
\section{Figures, graphiques et photos}

\subsection{TikZ}
% partie 4 : page 1
\begin{frame}
\frametitle{Créer un graphique}
Avec un peu de patience, on peut créer des graphiques très complexes à l'aide de TikZ.
\begin{center}
\begin{tikzpicture}
% horizontal axis
\draw[->] (0,0) -- (6,0) node[anchor=north] {$f/f_N$};
% labels
\draw   (0,0) node[anchor=north] {0}
                (2,0) node[anchor=north] {1}
                (4,0) node[anchor=north] {2};
% ranges
\draw   (1,3.5) node{{\scriptsize Constant flux}}
        (4,3.5) node{{\scriptsize Field weakening}};

% vertical axis
\draw[->] (0,0) -- (0,4) node[anchor=east] {$U_s,\varPsi_s$};
% nominal speed
\draw[dotted] (2,0) -- (2,4);

% Us
\draw[thick] (0,0) -- (2,2) -- (6,2);
\draw (1,1.5) node {$U_s$}; %label

% Psis
\draw[thick,dashed] (0,3) -- (2,3) parabola[bend at end] (6,1);
\draw (2.5,3) node {$\varPsi_s$}; %label

\end{tikzpicture}
\end{center}
\end{frame}

\subsection{Des photos}
% partie 4 : page 2
\begin{frame}
\frametitle{Insérer une image jpg}
Il est facile d'ajouter des figures ou des photos dans une présentation Beamer.
\begin{center}
\begin{tabular}{cc}
\includegraphics[width=3cm]{Bibendum.jpg} &
\includegraphics[width=6cm]{clumeq.jpg} \\
\end{tabular}
\end{center}
\end{frame}

% partie 4 : page 3
\begin{frame}
\only<1>{
\frametitle{Université Laval jadis}
\begin{center}
\includegraphics[width=8cm]{ul1.jpg}
\end{center}
}
\only<2>{
\frametitle{Université Laval maintenant}
\begin{center}
www.ulaval.ca\\[0.2cm]
\includegraphics[width=8cm]{ul2.jpg}
\end{center}
}
\end{frame}

\end{document}