\documentclass{beamer}

%\includeonlyframes{current}
% beamerthemesplit: barre de navigation, nom et titre au bas de la page
% beamerthemeclassic: plus sobre, barre de navigation mais pas de nom au bas de la page
%\usepackage{beamerthemesplit}
%\usepackage{beamerthemeclassic}
\usepackage[utf8x]{inputenc}
% Document en fran{\c c}ais, cesure, chapitres, etc
\usepackage[francais]{babel}
\usepackage{multimedia}
\usepackage{pgf}
\usepackage{tikz}
\mode<presentation> {

%  \setbeamertemplate{background canvas}[vertical shading][bottom=red!10,top=blue!10]f

%  \usetheme{Warsaw}
%\usetheme{JuanLesPins}
\usetheme{Darmstadt}
%\usetheme{classic}
  \setbeamercovered{transparent}
%\usecolortheme{crane}
%  \usefonttheme[onlysmall]{structurebold}
\definecolor{rouge}{RGB}{255,0,0}
\definecolor{or}{RGB}{255,204,0}
\definecolor{noir}{RGB}{0,0,0}
\setbeamercolor{structure}{fg=rouge,bg=or}
\setbeamercolor{titlelike}{fg=black,bg=or}
\setbeamercolor{frametitle}{fg=black,bg=or}
\setbeamercolor{item}{fg=rouge,bg=or}
}


\title{Modélisation mathématique \\(MAT-2420) }
\author{Votre nom}
\institute{D{\'e}pt. de math{\'e}matiques et de statistique\\
              Université Laval} 

\date{\today}
%{\'Equations diff\'erentielles de grande dimension\\
%   en sciences et en g\'enie}


\begin{document}



 \frame{

%\insertlogo{\includegraphics[height=1.0cm]{giref_2004.jpg}}
%\insertlogo{\includegraphics[height=1.5cm]{figures/giref_2004.jpg}}
\titlepage
}

\section[Plan]{}

\frame{\tableofcontents}

\section{I. Introduction}

\frame{ \frametitle{Renseignements généraux}
\begin{itemize}
  \item Beamer est le PowerPoint de Latex.
  \item La syntaxe est la même que dans Latex.
  \item Il faut de plus utiliser la commande «frame» qui est illustrée dans ce document.
  \item On trouvera la \href{http://www.ctan.org/tex-archive/macros/latex/contrib/beamer/doc/beameruserguide.pdf}{\alert{documentation}} en cliquant sur ce lien. 
\end{itemize}
}






\section{II. Transfert de chaleur}

 \frame{
 \frametitle{Équation de la chaleur}

 
 \[
     \rho c_p \frac{\partial T}{\partial t} - \nabla \cdot (K \nabla T) = 0
 \]
\begin{itemize}
  \item On complète avec des conditions initiales et aux limites.
  \item La constante $D = K/(\rho c_p)$ est la diffusion thermique.
  \item Les unités de $D$ sont des L$^2$T$^{-1}$ 
\end{itemize}

 
% 
 }



\end{document}

